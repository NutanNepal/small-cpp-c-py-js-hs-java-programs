\documentclass{article}
    \usepackage[utf8]{inputenc}
    
    \title{Complex Variables}
    \author{nnutannep}
    \date{October 2018}
    \usepackage{amsmath}
    \usepackage{natbib}
    \usepackage{graphicx}
    
    \begin{document}
    
    \maketitle
    
    \section{Residues and Poles}

    \paragraph{Example 1.}The function$$\frac{z + 1}{z^2(z^2+1)}$$
     has the three isolated singular points $z=0$ and $z=\pm i.$

    \paragraph{Example 2.} The origin is a singular point of the principal branch
    $$\text{Log $z$ = ln $r$}\, + i\Theta \quad (r > 0, -\pi < \Theta < \pi)$$
    of the logarithmic function. It is not, however an $isolated$ singular point
    since every deleted $\epsilon $ neighborhood of it contains points on the negative
    real axis and the branch is not even defined there.

    \paragraph{Example 3.}The function $$\displaystyle \frac{1}{\sin{(\pi/z)}}$$ has the 
    singular points $z=0$ and $z=1/n\, (n=\pm1,\pm2,...)$ all lying on the segment of the 
    real axis from $z=-1$ to $z=1$. Each singular point except $z=0$ is isolated. The 
    singular point $z=0$ is not isolated because every deleted $\epsilon$neighborhood 
    of the origin contains other singular points of the function. More precisely, when 
    a positive number $\epsilon$ is specified and $m$ is any positive integer such that 
    $m>1/\epsilon$, the fact that $0<1/m<\epsilon$ means that the point $z=1/m$ lies in 
    the deleted $\epsilon$ neighborhood $0<|z| < \epsilon$.
    
    \end{document}